\documentclass[conference,onecolumn]{IEEEtran}
\usepackage{enumitem}
\usepackage{cite}
\usepackage{graphicx}
\usepackage{float}
\graphicspath{{./images}}

\title{XM Radio Reception with the PlutoSDR}

\author{
\IEEEauthorblockN{Owen Sowatzke}
\IEEEauthorblockA{\textit{Electrical Engineering Department} \\
\textit{University of Arizona}\\
Tucson, USA \\
osowatzke@arizona.edu}
\and
\IEEEauthorblockN{Glenn Alan Walker}
\IEEEauthorblockA{\textit{Electrical Engineering Department} \\
\textit{University of Arizona}\\
Tucson, USA \\
gaw@arizona.edu}}

\begin{document}
\maketitle

% Proposal
% XM radio leverages a combination of satellites and terrestrial repeaters to diversify its transmitted signal. The satellites transmit QSPK-modulated symbols, and the terrestrial repeaters leverage COFDM modulation \cite{5586866}. For our final project, we propose demodulating the signal from a single satellite and performing forward error correction (FEC). Our project will leverage the time, frequency, and frame synchronization methods covered in class. Additionally, it will extend the material covered in class to forward error correction. XM radio is a proprietary signal, and the technical details are documented only in patents such as \cite{a2008_us8260192b2, marko_2012_us8667344b2}. Major milestones for our project include: performing timing and frequency synchronization, extracting the master frame preamble (MFP) and frame synchronization preamble (FSP), and demodulating the signal from a single satellite. Additional stretch goals that will be addressed only if time permits include: demodulating the COFDM signals from terrestrial repeaters, combining the returns from multiple satellites and/or the terrestrial repeaters, and playing XM channel 1 audio (free preview channel). We believe that this project will reinforce what we learned in the course and provide invaluable experience performing FEC and OFDM demodulation.

\section{Introduction}

In this project, we demodulate XM radio transmissions and perform forward error correction (FEC) with our Pluto SDR. XM radio uses a combination of satellites and terrestrial repeaters to diversify its transmitted signal. The satellites use QPSK modulation, and the terrestrial repeaters use coded orthogonal frequency domain multiplexing (COFDM). Our work focuses on the time-division multiplexed (TDM) signal from a single satellite. To demodulate this signal we leverage concepts from the course such as time synchronization, carrier synchronization, and frame synchronization. We also perform FEC, which was not covered in the course. With respect to FEC, we perform time deinterleaving, Virterbi decoding, and Reed Solomon Decoding. Because XM radio is a proprietary signal, it is documented only in patents such as \cite{a2008_us8260192b2, marko_2012_us8667344b2}. However, the patents leave out portions of the details, which require us to reverse-engineer portions of the system.

\iffalse
\begin{figure}[H]
	\centerline{\fbox{\includegraphics[width=0.5\textwidth]{xm_spectrum.png}}}
	\caption{XM Radio Spectrum \cite{a1999_us6724827b1}}
	\label{fig::xm_spectrum}
\end{figure}
\fi

\begin{figure}[H]
	\centerline{\fbox{\includegraphics[width=0.5\textwidth]{xm_satellite_config.png}}}
	\caption{XM Radio Satellite Configuration \cite{5586866}}
	\label{fig::xm_satellite_config}
\end{figure}

XM Radio divides their content across two separate ensembles (Ensemble A and Ensemble B), which are illustrated in Figure \ref{fig::xm_satellite_config}. Each ensemble uses transmitter diversity to improve signal quality and prevent dropouts. The diversity scheme specifically uses QPSK-modulated signals from two separate satellites and a terrestrial COFDM-modulated signal. We concentrate specifically on the TDM receiver. An example of its architecture is displayed in Figure \ref{fig::tdm_receiver}. 

\begin{figure}[H]
	\centerline{\fbox{\includegraphics[width=0.5\textwidth]{tdm_receiver.png}}}
	\caption{TDM Receiver Architecture \cite{a2008_us8260192b2}}
	\label{fig::tdm_receiver}
\end{figure}

\begin{figure}[H]
	\centerline{\fbox{\includegraphics[width=0.8\textwidth]{timing_carrier_sync.png}}}
	\caption{GNU Radio Flowchart That Implements Timing and Carrier Synchronization}
	\label{fig::timing_carrier_sync}
\end{figure}

The XM radio signal is transmitted through a square-root raised cosine filter, which limit the signal bandwidth [ADD A CITATION HERE]. Because of this, the receiver needs a matching square root raised cosine filter. The root-raised cosine filter is a type II Nyquist filter, which results in zero intersymbol interference when the received signal is sampled at the midpoint of each symbol period. When the PlutoSDR samples the received signal, this condition is not strictly enforced. This results in significant spreading in our constellation as illustrated in Figure \ref{fig::constellation_no_timing_comp}.

\begin{figure}[H]
	\centerline{\fbox{\includegraphics[width=0.5\textwidth]{constellation_no_timing_comp.png}}}
	\caption{Received Constellation Before Timing Synchronization}
	\label{fig::constellation_no_timing_comp}
\end{figure}

We can correct for this timing offset using a timing synchronization block. For this purpose, we use a GNU radio symbol sync block. This symbol sync block can be subdivided into 4 blocks: interpolator, timing error detector, loop filter, and controller. The interpolator applies a fractional delay. The timing error detector measures the timing offset. The loop filter stabilizes the process. And the controller manages the interpolation process. For our work, we specifically choose a Gardner Timing Error detector because it is robust to frequency errors. Our constellation after timing compensation is shown in Figure \ref{fig::constellation_after_timing_comp}.

\begin{figure}[H]
	\centerline{\fbox{\includegraphics[width=0.5\textwidth]{constellation_after_timing_comp.png}}}
	\caption{Received Constellation After Timing Synchronization}
	\label{fig::constellation_after_timing_comp}
\end{figure}

After timing compensation, we see that the amplitude of the constellation has stabilized. However, the resulting constellation is a ring. This occurs because after phase and frequency errors. The phase error leads to a static tilt of the constellation and the frequency offset causes our constellation to rotate. To correctly demodulate the signal, we must also remove the frequency offset in the signal. We do this in 2 stages: coarse frequency compensation and fine frequency compensation. We implement the coarse frequency compensation algorithm using a custom GNU radio python block. This block raises our received data to the fourth power to remove the QPSK modulation. Then, it takes an FFT of the resulting signal. The peak index of the FFT provides us with an estimate of the coarse frequency error. When solving for the error we also must divide the frequency error by 4 to account for raising the received data to the 4th power prior to the FFT. The FFT output and the corresponding frequency error for one frame of data is shown in Figure \ref{fig::cfo_frequency_estimate}. Examining the Figure, we observe a frequency error of roughly 15 kHz [CHECK SIGN].

\begin{figure}[H]
	\centerline{\fbox{\includegraphics[width=0.5\textwidth]{cfo_frequency_estimate.png}}}
	\caption{Coarse Frequency FFT Output}
	\label{fig::cfo_frequency_estimate}
\end{figure}

The coarse frequency compensation is limited by the FFT size and the rate at which the frequency drifts. We add an additional fine frequency compensation block to resolve the rest of the error. We use the GNU radio Costas Loop block for this analysis. The GNU radio Costas Loop closely resembles the timing synchronization block. It includes a phase error detector, a loop filter, a direct digital synthesizer and a phase rotator.
The phase rotator adjusts the phase of the received signal by multiplying it with a phasor. The phase error detector then detects the phase error, which is fed into a loop filter for stabilization. Finally the Direct Digital Synthesizer creates a coherent phasor which removes residual phase and frequency offsets. The constellation after fine frequency compensation is shown in Figure \ref{fig::constellation_after_fine_carrier_comp}.

\begin{figure}[H]
	\centerline{\fbox{\includegraphics[width=0.5\textwidth]{constellation_after_fine_carrier_comp.png}}}
	\caption{Received Constellation After Fine Carrier Synchronization}
	\label{fig::constellation_after_fine_carrier_comp}
\end{figure}

XM radio divides its transmitted data into frames. These frames are marked by two different preambles: the MFP (master frame preamble) and the FSP (fast synchronization preamble). The FSP marks the start of the data portion and corrects for ambiguities, while the MFP is used to align the signal from each satellite \cite{a2008_us8260192b2}. Both preambles and their relative timing are illustrated in Figure \ref{fig::tdm_frame_format}.

\begin{figure}[H]
	\centerline{\fbox{\includegraphics[width=0.8\textwidth]{tdm_frame_format.png}}}
	\caption{TDM Frame Format \cite{a2008_us8260192b2}}
	\label{fig::tdm_frame_format}
\end{figure}

The patents we referenced did not provide the MFP or FSP. As a result, we identified them ourselves using the auto-correlation of our signal. We considered the FSP first because it occurred more frequently in our collected data. To compute the FSP, we correlated an FSP duration of samples with a copy delayed by the FSP separation. By sweeping the starting index until we maximized the auto-correlation, we were able to effectively identify the start of our FSP. Our approach is illustrated in Figure \ref{fig::finding_fsp} and the best auto-correlation is shown in Figure \ref{fig::fsp_correlation}.

\begin{figure}[H]
	\centerline{\fbox{\includegraphics[width=0.5\textwidth]{finding_fsp.png}}}
	\caption{Algorithm for Indentifying FSP}
	\label{fig::finding_fsp}
\end{figure}

\begin{figure}[H]
	\centerline{\fbox{\includegraphics[width=0.5\textwidth]{fsp_correlation.png}}}
	\caption{Auto-Correlation of Optimum FSP Selection}
	\label{fig::fsp_correlation}
\end{figure}

We can also examine some of the properties of the FSP by looking at its constellation. In Figure \ref{fig::fsp_constellation}, we show the constellation of the FSP symbols and data symbols.

\begin{figure}[H]
	\centerline{\fbox{\includegraphics[width=0.5\textwidth]{fsp_constellation.png}}}
	\caption{Constellation of FSP Symbols vs Data Symbols}
	\label{fig::fsp_constellation}
\end{figure}

Examining the figure, we see that the FSP uses bi-phase modulation (phase-shifted BPSK) instead of QPSK modulation like the proceeding signals. To 
improve our correlation performance going forward, we demodulate each of the FSP constellation points. Note that our results are ambiguous by $180^{\circ}$, so we assume a leading 1 in the FSPs. Errors reported by the Reed Solomon decoder (when implemented) can be used to resolve this ambiguity. If we receive high error counts, we know to rotate the phase of our FSP by $180^{\circ}$.

We can perform a similar procedure to identify the MFP. However, to avoid false alarms from our FSP, we take advantage of the MFP located illustrated in Figure \ref{fig::tdm_frame_format}. We specifically know that the MFP will always be located right before the FSP. Therefore, we consider only MFP positions right before the FSP as illustrated in Figure \ref{fig::finding_mfp}.

\begin{figure}[H]
	\centerline{\fbox{\includegraphics[width=0.5\textwidth]{finding_mfp.png}}}
	\caption{Algorithm for Indentifying MFP}
	\label{fig::finding_mfp}
\end{figure}

The base case auto-correlation provides the location of the MFP. For our data, the best case auto-correlation is shown in Figure \ref{fig::mfp_correlation} and the resulting constellation is shown in Figure \ref{fig::mfp_constellation}.

\begin{figure}[H]
	\centerline{\fbox{\includegraphics[width=0.5\textwidth]{mfp_correlation.png}}}
	\caption{Auto-Correlation of Optimum MFP Selection}
	\label{fig::mfp_correlation}
\end{figure}

\begin{figure}[H]
	\centerline{\fbox{\includegraphics[width=0.5\textwidth]{mfp_constellation.png}}}
	\caption{Constellation of MFP Symbols vs Data Symbols}
	\label{fig::mfp_constellation}
\end{figure}

Examining the constellation, we see that the MFP is also bi-phase. To 
improve our correlation performance going forward, we demodulate each of the MFP constellation points. Note that our results are ambiguous by $180^{\circ}$ as described above. For now we assume a leading 1 in the MFPs.

% Add plot of FSP in constellation w/ different color

% Add picture of setup

\begin{itemize}
	\item Nyquist Filter
	\item Timing Synchronization
	\item Carrier Compensation
	\begin{itemize}
		\item Coarse
		\item Fine
	\end{itemize}
	\item Frame Synchronization
\end{itemize}

\nocite{5586866}
\nocite{a2008_us8260192b2}
\nocite{marko_2012_us8667344b2}
\nocite{collins_2018_softwaredefined}
\nocite{chaudhari_2022_timing}
\nocite{650240}
\bibliographystyle{IEEEtran}
\bibliography{sources}{}
%\bibliographystyle{ieeetr}
\end{document}