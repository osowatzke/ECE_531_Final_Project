\documentclass[conference,onecolumn]{IEEEtran}
\usepackage{enumitem}
\usepackage{cite}

\title{XM Radio Demodulation}

\author{
\IEEEauthorblockN{Owen Sowatzke}
\IEEEauthorblockA{\textit{Electrical Engineering Department} \\
\textit{University of Arizona}\\
Tucson, USA \\
osowatzke@arizona.edu}
\and
\IEEEauthorblockN{Glenn Alan Walker}
\IEEEauthorblockA{\textit{Electrical Engineering Department} \\
\textit{University of Arizona}\\
Tucson, USA \\
gaw@arizona.edu}}

\begin{document}
\maketitle

Perform single satellite demodulation as well a FEC decoding.

\begin{itemize}
\item Phase/Timing QPSK tracker to pull out the MFP
\begin{itemize}
\item Needed to de-interleave for the FEC
\end{itemize}
\item Demodulate/Decode QPSK XM signal from single satellite single ensemble
\item Demodulate/Decode multiple QPSK XM signals for a single ensemble
\item Demodulate/Decoced 2 satellites + 1 terrestrial ensemble
\item Play Channel 0 audio
\end{itemize}

Since XM is a proprietary signal, everything we do would be some sort of "discovery" as it isn't documented other than patents.  So just getting the FPS and MFP sequences are "new".

\nocite{a2008_us8260192b2}
\nocite{marko_2012_us8667344b2}
\nocite{alldatasheetcom_2015_sta400a}
\nocite{5586866}
\nocite{chaudhari_2022_timing}
\nocite{650240}
\nocite{collins_2018_softwaredefined}
\bibliographystyle{IEEEtran}
\bibliography{sources}{}
%\bibliographystyle{ieeetr}
\end{document}